\documentclass{article}

\usepackage{amsmath}

\usepackage{lmodern}

\usepackage{graphicx} 

\usepackage{fancyhdr}

\usepackage[margin=1.2in]{geometry} 

\setlength\headheight{26pt} 


\pagestyle{fancy}
\fancyhf{}
\rhead{Nicol�s Vald�s \\ FI3104-1 2018B \\ 27/09/18}
\lhead{\includegraphics[scale=0.52]{logo}}



\begin{document}
\text{} \vspace{0.3cm}
\begin{center}
\LARGE {\bf Tarea 1} 
\end{center}

\normalsize 

\section*{Problema 1}
\subsection*{Introducci�n} 

La definici�n de la derivada $f'(x)$ de una funci�n $f(x)$ es 
\begin{align}
f'(x)= \lim_{h\to 0} \frac{f(x+h)-f(x)}{h},
\end{align}
donde el l�mite por naturaleza es un proceso {\it continuo}. Claramente al calcular derivadas con un computador, esto no es factible. La opci�n m�s simple para calcular una derivada ser�a 
\begin{align}
f'(x)\approx \frac{f(x+h)-f(x)}{h},
\end{align} 
lo cual sale a partir de hacer una expansi�n de Taylor para $f$, y truncar t�rminos de orden mayor a $h$. La idea es tomar un $h$ peque�o para que esta aproximaci�n funcione. Si uno se queda con m�s t�rminos en la serie de Taylor, y juega con �lgebra, puede llegar a una aproximaci�n que trunca t�rminos de orden mayor a $h^3$. 
\begin{align}
d
\end{align}


\subsection*{Metodolog�a} 

\subsection*{Resultados} 
\begin{figure}[ht!]
\centering
\includegraphics[scale=0.8]{PlotP1_Float32}
\end{figure}
\begin{figure}[ht!]
\centering
\includegraphics{PlotP1_Float64}
\end{figure}
\begin{figure}[ht!]
\centering
\includegraphics{PlotP1_Float128}
\end{figure}
\subsection*{Conclusiones} 

\section*{Problema 2}
\subsection*{Introducci�n} 
\subsection*{Metodolog�a} 
\subsection*{Resultados} 
\subsection*{Conclusiones} 






\end{document} 